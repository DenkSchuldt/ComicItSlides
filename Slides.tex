\documentclass[utf8]{beamer}
\mode<presentation>
\usepackage[spanish]{babel}
\usepackage{multicol}
\useoutertheme{infolines} 
\usepackage{graphicx}
\usetheme{boxes} % other themes: AnnArbor, Antibes, Bergen, Berkeley, Berlin, Boadilla, boxes, CambridgeUS, Copenhagen, Darmstadt, default, Dresden, Frankfurt, Goettingen, Hannover, Ilmenau, JuanLesPins, Luebeck, Madrid, Maloe, Marburg, Montpellier, PaloAlto, Pittsburg, Rochester, Singapore, Szeged, classic

\author{Ana, Liliana, Denny}
\definecolor{lightblue}{rgb}{0,.5,1}
%\beamertemplateshadingbackground{lightblue!50}{lightblue!50}

%\setbeamercovered{transparent}
%\setbeamercovered{transparent}
\usebackgroundtemplate{\includegraphics[width= \paperwidth, height=\paperheight]{comicit1fondo3.jpg}}

\begin{document}
	\begin{frame}

		\frametitle{
			\color{blue}\textbf{\begin{center}{\Huge{¡Creando Historietas!}}\end{center}}
			%\newline Ana Arias, Liliana Ramos, Denny Schuldt
			%\color{red}\begin{center}--------------------------------o--------------------------------\end{center}
		}
		%\framesubtitle{\textbf{Autores:} Ana Arias, Liliana Ramos, Denny Schuldt
		%}
		\begin{center} 
				 \includegraphics[width=0.45\textwidth]{comicit.jpg} %Midifico el width para cambiar el tamaño%
		\end{center}
	\end{frame}
	\begin{frame}
		\frametitle{
			\color{red}\textbf{\begin{center}{\huge{Funcionalidades}}\end{center}}
			\color{blue}\textbf{\begin{center}{\huge{¡Haz que tu imagen pase de ser buena a ser 								increíble!}}\end{center}}}
			\begin{center}
			\begin{itemize}
			\item\textbf{ EDICIÓN DE IMAGENES:}
			\newline
			Una opción que permite colocar efectos a la imagen que  desees,
			\newline
			 entre estos efectos tenemos Sepia, Blanco y Negro;  modificar el 
			\newline
			brillo, contraste, entre otras características de la imagen.
			\end{itemize}
		\end{center} 
	\end{frame}
	\begin{frame}
		\begin{center} 
				 \includegraphics[width=0.45\textwidth]{images.jpg} %Midifico el width para cambiar el tamaño%
				 \includegraphics[width=0.45\textwidth]{4.jpg} %Midifico el width para cambiar el tamaño%
				 \newline				 
\includegraphics[width=0.45\textwidth]{5.jpg} %Midifico el width para cambiar el tamaño%
		\end{center}
	\end{frame}

	\begin{frame}
		\frametitle{
			\color{red}\textbf{\begin{center}{\huge{Funcionalidades}}\end{center}}
			\color{blue}\textbf{\begin{center}{\huge{Crea el diálogo de tus personajes}}\end{center}}}
			\begin{center}
			\begin{itemize}
			\item\textbf{LISTA DE BURBUJAS DE DIÁLOGO:}
			\newline
			Proporcionamos una lista con varios formatos de burbujas de
			\newline
			 diálogo, de las cuales el usuario puede escoger las que más  
			\newline
			se ajuste a su necesidad.
			\item\textbf{EDICIÓN DE BURBUJAS DE DIÁLOGO:}
			\newline
			El usuario podrá editar el texto dentro de la burbuja de diálogo 
			\newline
			que haya escogido, con la fuente, tamaño y color que desee.
			\end{itemize}
		\end{center} 
	\end{frame}	
	\begin{frame}
			\begin{center}
			\begin{itemize}
			\item\textbf{LISTA DE ICONOS:}
			\newline
			Proporcionamos una lista de iconos básicos y personalizados que
			\newline
			 permitirán al usuario crear imágenes más reales y divertidas.
			\newline
			\newline
			\item\textbf{EDICIÓN DE TEXTO O TEXTO PREDETERMINADO:}
			\newline
			Además de íconos, el usuario tendrá una lista de textos divertidos
			\newline
			para darle más creatividad a las escenas que esté creando. Si el u
			\newline
			 suario desea crear nuevos diseños tendrá la opción de hacer sus 
			\newline
			 propios dibujos y colocarlos en la imagen al igual que los iconos.
			\end{itemize}
		\end{center} 
	\end{frame}	
	\begin{frame}
	\begin{center}
		\includegraphics[width=0.45\textwidth]{6.jpg} %Midifico el width para cambiar el tamaño%
	\end{center}
	\end{frame}	

\end{document}

