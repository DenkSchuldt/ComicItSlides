\documentclass[utf8]{beamer}
\mode<presentation>
\usepackage[spanish]{babel}
\usepackage{multicol}
\useoutertheme{infolines} 
\usepackage{graphicx}
\usetheme{boxes} % other themes: AnnArbor, Antibes, Bergen, Berkeley, Berlin, Boadilla, boxes, CambridgeUS, Copenhagen, Darmstadt, default, Dresden, Frankfurt, Goettingen, Hannover, Ilmenau, JuanLesPins, Luebeck, Madrid, Maloe, Marburg, Montpellier, PaloAlto, Pittsburg, Rochester, Singapore, Szeged, classic

\author{Ana, Liliana, Denny}
\definecolor{lightblue}{rgb}{0,.5,1}
%\beamertemplateshadingbackground{lightblue!50}{lightblue!50}

\usebackgroundtemplate{\includegraphics[width= \paperwidth, height=\paperheight]{comicit1fondo3.jpg}}

\begin{document}
	\begin{frame}
		\frametitle{
			\color{blue}\textbf{\begin{center}{\Huge{¡Creando Historietas!}}\end{center}}
			%\newline Ana Arias, Liliana Ramos, Denny Schuldt
			%\color{red}\begin{center}--------------------------------o--------------------------------\end{center}
		}
		%\framesubtitle{\textbf{Autores:} Ana Arias, Liliana Ramos, Denny Schuldt
		%}
		\begin{center} 
				 \includegraphics[width=0.45\textwidth]{comicit.jpg} %Midifico el width para cambiar el tamaño%
		\end{center}
	\end{frame}
	\begin{frame}	
	 	\frametitle{
			\color{black}\textbf{\begin{center}{\huge{Autores}}\end{center}}
			
		}
		   \begin{center} 
			
			- Ana Arias
			\newline
			- Liliana Ramos
			\newline
			- Denny Schuldt
			\newline
			\newline
			Lenguajes de Programación
			\\2012 - II Término
			
		\end{center} 
			
			
	\end{frame}
	\begin{frame}
		\textbf{Funcionalidades}
		\newline
		\begin{itemize}
			\item Funcionalidad 1
			\newline
			Hace x1 cosa con y1 parte de z1 dispositivo...
			\pause
			\item Funcionalidad 2
			\newline
			Hace x2 cosa con y2 parte de z2 dispositivo...
			\pause
			\item Funcionalidad 3
			\newline
			Hace x3 cosa con y3 parte de z3 dispositivo...
			\pause
			\newline
			\newline
			\newline
			\flushright
			...Wow!
		\end{itemize}
	\end{frame}	
	\begin{frame}
		\textbf{Como funciona?}
		\newline
		\newline
		%Colocar aqui el modo de funcionamiento esperado.%
		sdfvghaj  qfhnwjglj fhban yvawjhbs hteggdj ushf sdfvghaj  qfhnwjglj fhban yvawjhbs hteggdj ushf
		sdfvghaj  qfhnwjglj fhban yhbs hteggdj ushf sdfvghaj  qfhnwjglj fhban yvawjhbs hteggdj ushf
		sdfaj  qfhnwjglj fhban yvawjhbs hteggdj ushf.
		sdfvghaj  qfhnwjglj fhbahbs hteggdj ushf sdfvghaj  qfhnwjglj fhban yvawjhbs hteggdj ushf
		sdfvghaj  qfhnwjglj fhban yvahbs hteggdj ushf sdfvghaj  qfhnwjglj fhban yvawjhbs hteggdj ushf
		sdfvghaj  qfhnwjglj fhban yvawhbs hteggdj ushf sdfvghaj  qfhnwjglj fhban yvawjhbs hteggdj ushf
		hteggdj ushf sdfvghaj  qfhnwjglj fhban yvawjhbs hteggdj ush.
		%Hasta aquí la parte de funcionamiento.%
	\end{frame}
	\begin{frame}
		\textbf{Ejemplos de plantillas:}
		\newline	
		\begin{center} 
			\includegraphics[width=0.8\textwidth]{plantillas.jpg}
		\end{center}
	\end{frame}
	\begin{frame}
		\transdissolve
		 \includegraphics[width=1\textwidth]{failure.jpg} %Midifico el width para cambiar el tamaño%
	\end{frame}
\end{document}

